\documentclass[Royal,times,sageh]{sagej}

\usepackage{moreverb,url,natbib, multirow, tabularx}
\usepackage[colorlinks,bookmarksopen,bookmarksnumbered,citecolor=red,urlcolor=red]{hyperref}



% tightlist command for lists without linebreak
\providecommand{\tightlist}{%
  \setlength{\itemsep}{0pt}\setlength{\parskip}{0pt}}





\begin{document}


\setcitestyle{aysep={,}}

\title{Análisis de los determinantes en la fecundidad de las mujeres en
Bolivia}

\runninghead{Uthor \emph{et al}.}

\author{Valentina Valdez Vega*\affilnum{1}}

\affiliation{\affilnum{1}{Estudiante de la carrera de ``Economía e
Inteligencia de Negocios'', Universidad Católica Boliviana ``San
Pablo''}}

\corrauth{Corresponding author name, This is sample corresponding
address.}

\email{\href{mailto:correspondingauthor@protonmail.com}{\nolinkurl{correspondingauthor@protonmail.com}}}

\begin{abstract}
Lorem ipsum dolor sit amet, consectetur adipiscing elit. Aenean ut elit
odio. Donec fermentum tellus neque, vitae fringilla orci pretium vitae.
Fusce maximus finibus facilisis. Donec ut ullamcorper turpis. Donec ut
porta ipsum. Nullam cursus mauris a sapien ornare pulvinar. Aenean
malesuada molestie erat quis mattis. Praesent scelerisque posuere
faucibus. Praesent nunc nulla, ullamcorper ut ullamcorper sed, molestie
ut est. Donec consequat libero nisi, non semper velit vulputate et.
Quisque eleifend tincidunt ligula, bibendum finibus massa cursus eget.
Curabitur aliquet vehicula quam non pulvinar. Aliquam facilisis tortor
nec purus finibus, sit amet elementum eros sodales. Ut porta porttitor
vestibulum. Integer molestie, leo ut maximus aliquam, velit dui iaculis
nibh, eget hendrerit purus risus sit amet dolor. Sed sed tincidunt ex.
Curabitur imperdiet egestas tellus in iaculis. Maecenas ante neque,
pretium vel nisl at, lobortis lacinia neque. In gravida elit vel
volutpat imperdiet. Sed ut nulla arcu. Proin blandit interdum ex sit
amet laoreet. Phasellus efficitur, sem hendrerit mattis dapibus, nunc
tellus ornare nisi, nec eleifend enim nibh ac ipsum. Aenean tincidunt
nisl sit amet facilisis faucibus. Donec odio erat, bibendum eu imperdiet
sed, gravida luctus turpis.
\end{abstract}

\keywords{Fecundidad, Bolivia, Regresión Logit, Regresión de Poisson;}

\maketitle

\section{Introducción}\label{introducciuxf3n}

Según los resultados de la última ``Encuesta de Demografía y Salud''
(EDSA) realizada por el Instituto Nacional de Estadística (INE), se
observó una disminución en la tasa de fecundidad llegando a 2,1 en
comparación a una tasa de 2,9 que se obtuvo en la pasada encuesta EDSA
de 2016. Esta caída en la fecundidad no es una sorpresa para países
dentro de la región latinoamericana y puede ser relacionada con cambios
culturales, nuevas expectativas de vida por parte de la población joven
y la creciente inserción al mercado laboral de la población femenina.

De esta forma, resulta necesario analizar los factores determinantes en
la fecundidad de las mujeres en Bolivia, para así poder entender de
mejor forma el porqué de esta disminución.

\section{Objetivos}\label{objetivos}

\subsection{Objetivo General}\label{objetivo-general}

Analizar los factores determinantes en la fecundidad de las mujeres
adultas y el número de hijos nacidos vivos por mujer en edad fértil en
Bolivia, a través de modelos lineales generalizados.

\subsection{Objetivos Específicos}\label{objetivos-especuxedficos}

\begin{itemize}
\item
  Desarrollar el modelo de regresión logístico para explicar los
  determinantes de la fecundidad en las mujeres adultas en Bolivia a
  partir de los datos obtenidos en la encuesta EDSA 2023.
\item
  Desarrollar el modelo de regresión de Poisson para modelar el número
  total de hijos nacidos vivos de las mujeres en edad fértil a partir de
  los datos obtenidos en la encuesta EDSA 2023.
\item
  Evaluar el efecto de los determinantes socioeconómicos y demográficos
  tales como el nivel de pobreza, educación y lugar de residencia en la
  fecundidad de las mujeres adultas y el número de hijos nacidos vivos
  de las mujeres en edad fértil en Bolivia.
\item
  Emplear pruebas de bondad de ajuste convencionales, como ser la
  devianza y el criterio de información de Akaike (AIC), para la
  evaluación de los modelos lineales generalizados utilizados en la
  presente investigación.
\end{itemize}

\section{Motivación}\label{motivaciuxf3n}

\section{Marco Teórico/ Revisión de
literatura}\label{marco-teuxf3rico-revisiuxf3n-de-literatura}

\section{Descripción del dataset}\label{descripciuxf3n-del-dataset}

Los datos que se usaron para trabajar con el modelado tanto de la
regresión logística como de la regresión de Poisson fueron obtenidos por
la última versión de la Encuesta de Demografía y Salud (EDSA), realizada
por el INE en 2023. La encuesta EDSA forma parte de las investigaciones
que se realizan de forma periódica y a nivel nacional, con el fin de
proporcionar información para el cálculo referente a los principales
indicadores de salud y demografía como ser la fecundidad, salud materna
e infantil, mortalidad infantil y de niñez, vacunación, estado
nutricional de los menores de seis años y anticoncepción.
Posteriormente, esta información se convierte en el pilar fundamental
para formular y evaluar el diseño de políticas públicas y programas que
se implementen en Bolivia bajo el Plan de Desarrollo Económico y Social
(PDES).

\section{Metodología}\label{metodologuxeda}

\section{Resultados y análisis}\label{resultados-y-anuxe1lisis}

\section{Conclusiones y
recomendaciones}\label{conclusiones-y-recomendaciones}

\section{Bibliography}\label{bibliography}

Link a \texttt{.bib} document via the YAML header, and bibliography will
be printed at the very end (as usual). The default bibliography style is
provided by Wiley as in \texttt{WileyNJD-AMA.bst}, do not delete that
file.

Use the Rmarkdown equivalent of the \LaTeX citation system using
\texttt{{[}@\textless{}name\textgreater{}{]}}. Example:
\citep{Taylor1937}, \citep{Knupp1999, Kamm2000}.

To include all citation from the \texttt{.bib} file, add
\texttt{\textbackslash{}nocite\{*\}} before the end of the document.

\bibliographystyle{sageh}
\bibliography{biblio}


\end{document}
