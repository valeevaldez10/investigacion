\documentclass[Royal,times,sageh]{sagej}

\usepackage{moreverb,url,natbib, multirow, tabularx}
\usepackage[colorlinks,bookmarksopen,bookmarksnumbered,citecolor=red,urlcolor=red]{hyperref}



% tightlist command for lists without linebreak
\providecommand{\tightlist}{%
  \setlength{\itemsep}{0pt}\setlength{\parskip}{0pt}}





\begin{document}


\setcitestyle{aysep={,}}

\title{Análisis de los determinantes en la fecundidad de las mujeres en
Bolivia}

\runninghead{}

\author{Valentina Valdez Vega\affilnum{1}}

\affiliation{\affilnum{1}{Estudiante de la carrera de ``Economía e
Inteligencia de Negocios'', Universidad Católica Boliviana ``San
Pablo''}}



\begin{abstract}
hola
\end{abstract}

\keywords{Fecundidad, Bolivia, Regresión Logit, Regresión Poisson;}

\maketitle

\section{Introducción}\label{introducciuxf3n}

Según los resultados de la última Encuesta de Demografía y Salud (EDSA
2023), realizada por el Instituto Nacional de Estadística (INE), se
observó una disminución en la tasa de fecundidad en Bolivia. La tasa de
fecundidad en 2023 llegó a 2,1 en comparación a la tasa de 2,9 que se
obtuvo en la pasada encuesta EDSA de 2016. Esta caída en la fecundidad
no es una sorpresa para países dentro de la región latinoamericana y
puede ser relacionada con cambios culturales, nuevas expectativas de
vida por parte de la población joven y la creciente inserción al mercado
laboral de la población femenina. Por otra parte, se ha visto una fuerte
asociación respecto a elevados niveles de pobreza, especialmente en
hogares monoparentales o rurales (UNPD, 2015), y lo que se entiende como
una alta fecundidad\footnote{Según la definición por el UNPD (2015), se
  define alta fecundidad como haber tenido cuatro o más hijos nacidos
  vivos.}.

La Encuesta de Demografía y Salud (EDSA) constituye una fuente de
información relevante al momento de obtener indicadores de salud y
demografía a nivel nacional, que posteriormente ayudarán el diseño de
políticas públicas implementadas en el país boliviano. La EDSA ofrece
información importante para analizar la fecundidad en Bolivia, ya que
permite vincular los patrones reproductivos con variables
sociodemográficas, culturales y de salud, además de gozar de una
cobertura a nivel nacional. De esta forma, esta encuesta actúa como
fuente de información esencial para la construcción de modelos
predictivos que identifican los factores más significativos asociados a
la fecundidad de mujeres adultas en edad fértil \footnote{Según la
  definición de la EDSA (2023) considera una mujer adulta en edad fértil
  a aquellas mujeres entre 20 y 49 años.} (INE \& UNFPA, 2023).

En este sentido, resulta necesario analizar los factores determinantes
en la fecundidad de las mujeres en Bolivia, para así poder entender de
mejor forma por qué existe una tendencia a la baja respecto a la tasa de
fecundidad y el número de hijos por mujer. Para tal efecto, es posible
utilizar modelos estadísticos de regresión tales como la regresión
logística y la regresión Poisson. Por un lado, la regresión logística
permite estimar la probabilidad de ocurrencia de una variable binaria a
partir de variables independientes, lo cual resulta clave para
identificar factores de fecundidad. Por otro lado, la regresión Poisson
permite examinar cómo varía la fecundidad en cuanto al número de hijos
según características individuales como la edad, nivel educativo,
actividad económica o pertenencia étnica (Schultz, 2006).

\section{Objetivos}\label{objetivos}

\subsection{Objetivo General}\label{objetivo-general}

Analizar los factores determinantes en la fecundidad de las mujeres
adultas y el número de hijos nacidos vivos por mujer adulta en edad
fértil en Bolivia, a través de modelos de regresión logit y Poisson.

\subsection{Objetivos Específicos}\label{objetivos-especuxedficos}

\begin{itemize}
\item
  Desarrollar el modelo de regresión logístico para explicar los
  determinantes de la fecundidad en las mujeres adultas en Bolivia a
  partir de los datos obtenidos en la encuesta EDSA 2023.
\item
  Desarrollar el modelo de regresión logístico para explicar los
  determinantes de la alta fecundidad en las mujeres adultas en Bolivia
  a partir de los datos obtenidos en la encuesta EDSA 2023.
\item
  Desarrollar el modelo de regresión de Poisson para modelar el número
  total de hijos nacidos vivos de las mujeres adultas en edad fértil a
  partir de los datos obtenidos en la encuesta EDSA 2023.
\item
  Evaluar el efecto de los determinantes socioeconómicos y demográficos
  tales como el nivel de pobreza, educación y lugar de residencia en la
  fecundidad de las mujeres adultas y el número de hijos nacidos vivos
  de las mujeres en edad fértil en Bolivia, basado en el enfoque de
  Schultz (Schultz, 2006).
\item
  Emplear pruebas de bondad de ajuste convencionales, como ser la
  devianza y el criterio de información de Akaike (AIC), para la
  evaluación de los modelos desarrollados en la presente investigación.
\end{itemize}

\section{Motivación}\label{motivaciuxf3n}

La fecundidad constituye uno de los indicadores demográficos más
relevantes dentro de la dinámica de un país, ya que influye de forma
directa en el crecimiento de una población, la estructura por edades y,
como consecuencia, permite diseñar políticas sociales y económicas como
sean requeridas. Como menciona Schultz (2006), el estudio de la
fecundidad no solo se reserva a una decisión propia de una familia o
persona, si no que tiene fuertes implicaciones económicas y sociales
dentro de una sociedad. A partir de esto, se refuerza la influencia de
la fecundidad como un factor esencial del desarrollo de un país y la
importancia de comprender los factores que determinan la fecundidad
dentro de una economía, para poder entender su comportamiento.

En Bolivia, tomando en cuenta los resultados presentados en la Encuesta
de Demografía y Salud, se ha observado que persisten diferencias
geográficas, étnicas y socioeconómicas que inciden en los patrones de
fecundidad de las mujeres. Como se observa en la figura 1 y se mencionó
en la sección 1, los resultados posteriores a la recolección de
información para la EDSA (en sus dos versiones 2016 y 2023) muestran una
tendencia a la baja en lo que compete a la tasa global de
fecundidad\footnote{La tasa global de fecundidad se entiende como el
  número promedio de hijos que tendrá una mujer a lo largo de su vida
  fértil, comprendida entre los 15 a 49 años.} a nivel Bolivia. En ambos
casos, se observa una disminución de 3,5 a 2,9 para el año 2016, y una
reducción a 2,1 para el año 2023. Sin embargo, comparando con otros
países de América Latina, se observa que desde el 2008 la tasa global de
fecundidad boliviana se mantiene mayor respecto a países como Ecuador,
Brasil y Chile. Aunque, como región, en todos los países se observa una
tendencia decreciente. Esta reducción en la tasa global de fecundidad
conduce a una población con una proporción menor de jóvenes, lo cual
genera desafíos para la fuerza laboral, la sostenibilidad de los
sistemas sociales y afecta directamente el crecimiento poblacional.

\section{Marco Teórico/ Revisión de
literatura}\label{marco-teuxf3rico-revisiuxf3n-de-literatura}

\section{Descripción del dataset}\label{descripciuxf3n-del-dataset}

\section{Metodología}\label{metodologuxeda}

\section{Resultados y análisis}\label{resultados-y-anuxe1lisis}

\section{Conclusiones y
recomendaciones}\label{conclusiones-y-recomendaciones}

\bibliographystyle{sageh}
\bibliography{biblio}


\end{document}
